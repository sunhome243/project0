\documentclass{article}
\usepackage{amsmath}
\usepackage{amssymb}

\title{Proof: Cardinality of Power Sets}
\author{Sunho Kim}
\date{Sep 8, 2025}

\begin{document}

\maketitle

\section{Problem Statement}
Prove inductively that a set $S$ with cardinality $n \geq 1$ has exactly $2^n$ unique subsets.

\section{Proof}

Let $P(n)$ be the statement: a set $S$ with cardinality $n \geq 1$ has exactly $2^n$ unique subsets.

\subsection{Base Case}
For $n = 1$, a set $S$ with cardinality 1 has $2^1 = 2$ unique subsets: $\emptyset, \{a\}$. Thus, the base case holds for $n = 1$.

\subsection{Inductive Case}
\textbf{Inductive Hypothesis:} Assume $P(k)$ holds for an arbitrary natural number $k$, which means a set $S$ with cardinality $k$ has exactly $2^k$ unique subsets.

\textbf{Inductive Step:} We want to show that a set $S$ with cardinality $k+1$ has exactly $2^{k+1}$ unique subsets.

When we add one more element (let this new element be $x$) to set $S$ that has cardinality $k$, we can form new subsets in the following way: for all original subsets of $S$, we can create a new subset by including the new element $x$. This means that for each of the original subsets of $S$, we can create a corresponding new subset by adding $x$.

Since there are $2^k$ original subsets of $S$ (by the inductive hypothesis), we can create $2^k$ new subsets by adding the element $x$ to each of those original subsets. This results in a total of $2^k + 2^k = 2 \cdot 2^k = 2^{k+1}$ unique subsets.

\section{Conclusion}
Since both the base case and inductive case hold, by mathematical induction we conclude that for all $n \geq 1$, a set $S$ with cardinality $n$ has exactly $2^n$ unique subsets.

\end{document}