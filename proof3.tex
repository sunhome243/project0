\documentclass{article}

\title{Proof 3}
\author{Flynn}
\date{}

\begin{document}

\maketitle

\section{Question}
Prove inductively that the complement of the union of any $n$ sets $S_1, S_2, \ldots, S_n$ is equivalent to the intersection of each of their individual complements (i.e., that $\overline{S_1 \cup S_2 \cup \cdots \cup S_n} = \overline{S_1} \cap \overline{S_2} \cap \cdots \cap \overline{S_n}$) for all $n \geq 1$.

\textbf{Hint:} it may be helpful to remember De Morgan's Law: $\overline{S \cup T} = \overline{S} \cap \overline{T}$

\section{Proof by Mathematical Induction}

For any collection of sets $S_1, S_2, \ldots, S_n$, we want to show that 
$$\overline{S_1 \cup S_2 \cup \cdots \cup S_n} = \overline{S_1} \cap \overline{S_2} \cap \cdots \cap \overline{S_n}$$
for all positive integers $n \geq 1$.

We proceed by mathematical induction on $n$.

\subsection{Base Case ($n = 1$)}
When we have only one set, we need to show that $\overline{S_1} = \overline{S_1}$.

This statement is clearly true by the reflexive property of equality, as any set is equal to itself.

\subsection{Base Case ($n = 2$)}
For the case where we have two sets, we need to demonstrate that $\overline{S_1 \cup S_2} = \overline{S_1} \cap \overline{S_2}$.

This equality is precisely the statement of De Morgan's Law for two sets, which is given as our foundational rule. Therefore, our base case for $n = 2$ holds true.

\subsection{Inductive Hypothesis}
Let us assume that the statement is true for some arbitrary positive integer $n = k$, where $k \geq 2$. That is, we assume that for any collection of $k$ sets, the following equality holds:
$$\overline{S_1 \cup S_2 \cup \cdots \cup S_k} = \overline{S_1} \cap \overline{S_2} \cap \cdots \cap \overline{S_k}$$

\subsection{Inductive Step}
Now we must prove that if the statement is true for $n = k$, then it must also be true for $n = k + 1$. We need to establish that:
$$\overline{S_1 \cup S_2 \cup \cdots \cup S_k \cup S_{k+1}} = \overline{S_1} \cap \overline{S_2} \cap \cdots \cap \overline{S_k} \cap \overline{S_{k+1}}$$

Starting with the left-hand side of our desired equality, we can rewrite the union of $k+1$ sets by grouping the first $k$ sets together:

$$\overline{S_1 \cup S_2 \cup \cdots \cup S_k \cup S_{k+1}} = \overline{(S_1 \cup S_2 \cup \cdots \cup S_k) \cup S_{k+1}}$$

This regrouping is valid due to the associative property of set union operations.

Next, we can apply De Morgan's Law for two sets to this expression, treating $(S_1 \cup S_2 \cup \cdots \cup S_k)$ as our first set and $S_{k+1}$ as our second set:

$$= \overline{(S_1 \cup S_2 \cup \cdots \cup S_k)} \cap \overline{S_{k+1}}$$

Now we can apply our inductive hypothesis to the first term $\overline{(S_1 \cup S_2 \cup \cdots \cup S_k)}$, since we assumed the statement holds for $k$ sets:

$$= (\overline{S_1} \cap \overline{S_2} \cap \cdots \cap \overline{S_k}) \cap \overline{S_{k+1}}$$

Finally, we can remove the parentheses using the associative property of set intersection:

$$= \overline{S_1} \cap \overline{S_2} \cap \cdots \cap \overline{S_k} \cap \overline{S_{k+1}}$$

This is exactly what we wanted to prove for the case $n = k + 1$.

\subsection{Conclusion}
Since we have established that the statement holds for our base cases ($n = 1$ and $n = 2$), and we have shown that if the statement is true for any positive integer $k$, then it must also be true for $k + 1$, we can conclude by the principle of mathematical induction that:
$$\overline{S_1 \cup S_2 \cup \cdots \cup S_n} = \overline{S_1} \cap \overline{S_2} \cap \cdots \cap \overline{S_n}$$
holds for all positive integers $n \geq 1$. This completes our proof of the generalized De Morgan's Law.

\end{document}
