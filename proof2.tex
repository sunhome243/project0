\documentclass{article}
\usepackage{amsmath}
\usepackage{amssymb}

\title{Proof: Subsets of Cardinality 2}
\author{Sunho Kim}
\date{Sep 8, 2025}

\begin{document}

\maketitle

\section{Problem Statement}
Prove inductively that a set $S$ with cardinality $n \geq 2$ has exactly $\frac{n(n-1)}{2}$ unique subsets of cardinality 2.

\section{Proof}

Let $P(n)$ be the statement: a set $S$ with cardinality $n \geq 2$ has exactly $\frac{n(n-1)}{2}$ unique subsets of cardinality 2.

\subsection{Base Case}
For $n = 2$, we have $\frac{2(2-1)}{2} = \frac{2 \cdot 1}{2} = 1$. 

A set of cardinality 2 has exactly one subset of cardinality 2: the set itself. Therefore, the base case holds for $n = 2$.

\subsection{Inductive Case}
\textbf{Inductive Hypothesis:} Assume $P(k)$ holds for an arbitrary natural number $k \geq 2$, which means a set $S$ with cardinality $k$ has exactly $\frac{k(k-1)}{2}$ unique subsets of cardinality 2.

\textbf{Inductive Step:} We want to show that a set $S$ with cardinality $k+1$ has exactly $\frac{(k+1)k}{2}$ unique subsets of cardinality 2.

When we add one more element (call it $x$) to a set $S$ with cardinality $k$, we create new subsets of cardinality 2 by pairing this new element $x$ with each of the $k$ original elements. This gives us exactly $k$ new subsets of cardinality 2.

Therefore, the total number of subsets of cardinality 2 is:
$$\frac{k(k-1)}{2} + k = \frac{k(k-1) + 2k}{2} = \frac{k(k-1+2)}{2} = \frac{k(k+1)}{2} = \frac{(k+1)k}{2}$$

Thus, the inductive case holds.

\section{Conclusion}
Since both the base case and inductive case hold, by mathematical induction we conclude that for all $n \geq 2$, a set $S$ with cardinality $n$ has exactly $\frac{n(n-1)}{2}$ unique subsets of cardinality 2.

\end{document} 